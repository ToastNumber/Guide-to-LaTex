\section{Environments}
\begin{tabular}{|c|p{8.9cm}|}
\hline

\textbf{Environment} & \textbf{Description}

\\
\hline
\verb£equation£ & The content of this environment will be a numbered equation.

\\
\hline
\verb£enumerate£ & Each \textbackslash{}item will be enumerated.

\\
\hline
\verb£itemize£ & Each \textbackslash{}item will be bullet-pointed

\\
\hline
\verb£description£ & The first word of each \textbackslash{}item is in bold.

\\
\hline
\verb£flushleft£ & Left-aligns the content.

\\
\hline
\verb£flushright£ & Right-aligns the content.

\\
\hline
\verb£center£ & Centers the content.

\\
\hline
\verb£quote£ & Represents the content as a quote.

\\
\hline
\verb£quotation£ & This is useful for longer quotes going over several paragraphs, because it indents the first line of each paragraph.

\\
\hline
\verb£verse£ & This is useful for poems where the line breaks are important. The lines are separated by issuing a \textbackslash{}\textbackslash{} at the end of a line and an empty line after each verse.

\\
\hline
\verb£verbatim£ & The content is displayed exactly as in the editor. This can be replicated in paragraph by using the command \verb£\verb£.
For example:

\verb£\verb#This is in verbatim#£

Will print \verb£This in in verbatim£.
\\
\hline
\end{tabular}